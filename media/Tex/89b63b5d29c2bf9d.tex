\documentclass[preview]{standalone}
\usepackage[english]{babel}
\usepackage{amsmath}
\usepackage{amssymb}
\begin{document}
\begin{align*}
F_{r}(\omega)&=\int_{-\infty}^{\infty}\sin(5\pi t)\cos{(\omega t) \ dt}
\end{align*}
\end{document}